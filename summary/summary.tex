\documentclass[a4paper, 11pt]{article}
% For UTF-8 encoding
\usepackage[utf8]{inputenc}
% For proper hyphenation etc.
\usepackage{babel}
% For clickable links
\usepackage{hyperref}
% Set page margins
\usepackage[top=2cm, bottom=2cm, left=2.5cm, right=2.5cm]{geometry}
\usepackage{amssymb}
\usepackage{amsmath}
\usepackage{tikz}
\usepackage{pgfplots}
\usetikzlibrary{calc,shadings}
\usetikzlibrary{positioning}
\usepackage{amsthm}
\usepackage{color}
\usepackage{algorithm,algorithmic}
\usepackage[singlespacing]{setspace}

\usepackage{todonotes}

%Definitions
\newtheorem{mydef}{Definition}
\newtheorem*{remark}{Remark}
\newtheorem{example}{Example}
\newtheorem{lemma}{Lemma}
\newtheorem{theorem}{Theorem}
\newtheorem{corollary}{Corollary}
\newtheorem{proposition}{Proposition}

\newcommand{\name}{BDD-Voodoo}

\makeatletter
\renewenvironment{quotation}
{\list{}{\listparindent=1.5em
		\itemindent=0pt
		\parsep\z@ \@plus\p@}%
	\item\relax}
{\endlist}
\makeatother

\newenvironment{customlegend}[1][]{%
	\begingroup
	% inits/clears the lists (which might be populated from previous
	% axes):
	\csname pgfplots@init@cleared@structures\endcsname
	\pgfplotsset{#1}%
}{%
	% draws the legend:
	\csname pgfplots@createlegend\endcsname
	\endgroup
}%

%definitions
\def\addlegendimage{\csname pgfplots@addlegendimage\endcsname}
% definition to insert numbers
\pgfkeys{/pgfplots/number in legend/.style={%
		/pgfplots/legend image code/.code={%
			\node at (0.295,-0.0225){#1};
		},%
	},
}

\usepackage{fancyhdr}

\newcommand{\HRule}[1]{\rule{\linewidth}{#1}}
\setcounter{tocdepth}{5}
\setcounter{secnumdepth}{5}

%-------------------------------------------------------------------------------
% TITLE PAGE
%-------------------------------------------------------------------------------

\begin{document}
	
\title{ \normalsize \textsc{Summary: }
	\\ [2.0cm]
	\HRule{0.5pt} \\
	\LARGE \textbf{\uppercase{Learning from gene expression regions within mouse brains}
		\HRule{1pt} \\ [0.5cm]
		\normalsize September 30, 2020 \vspace*{5\baselineskip}}
	
	\date{Summer semester 2020}
	
	\author{
		Tilman Hinnerichs \\
		Matrikelnummer: 4643427 \\ 
		Technische Universität Dresden\vspace{1cm}\\
		Tutor: Dr. Nico Scherf (MPI: CBS)}}
\maketitle
\newpage
\begin{abstract}
	
\end{abstract}

\tableofcontents

\newpage

\section{Possible prediction tasks}
\begin{itemize}
	\item predict gene expression for a given single structure
	\item predict structure from gene expression pattern
	\item predict structure form gene expression and image
	\item predict cancer type from morphology/pathologic image of cancer
	\item simulate loss of function/expression by removing one node of graph
\end{itemize}

\section{Predict gene expression values for single structures}
Predict gene expression per section/structure:\\
Take region as input and predict gene expression\\
Challenges:
\begin{itemize}
	\item how to normalize expression intensity (see discussion in DeepMOCCA paper, Sara Alghamdi), as there are regions with much more activity than others (e.g. bone narrow vs. bone boarder); thresholds for intensity varies across genes
	 
	\begin{itemize}
		\item over all intensities $\rightarrow$
		\item per structure $\rightarrow$  
		\item per gene $\rightarrow$ 
	\end{itemize}
	\item transfer learning working for other structure/regions
	\item dataset: Allen Mouse brain atlas vs. 
	\begin{itemize}
		\item \href{https://www.har.mrc.ac.uk/harwell-news/phenoview-new-tool-compare-impc-data/}{phenoview impc data}
		
		\item \href{https://www.mousephenotype.org/}{mousephenotype}
		
		\item \href{http://www.informatics.jax.org/expression.shtml}{HPO/MP project expression data}
	\end{itemize}
	\item structure specific features?
\end{itemize}

predict structure from gene expression pattern\\


\section{Pathology prediction}
predict structure from gene expression and images\\
\begin{itemize}
	\item take 
\end{itemize}

\end{document}