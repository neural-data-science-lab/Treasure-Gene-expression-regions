\documentclass[]{article}
\usepackage[left=3cm,right=3cm,top=1.5cm,bottom=2cm,includeheadfoot]{geometry} 
\usepackage{babel}
\usepackage{hyperref}
\usepackage{mwe}
\usepackage[markcase=noupper
]{scrlayer-scrpage}
\usepackage{amsmath}
\usepackage{amssymb}
\usepackage{multirow}

% bibliography natbib
\usepackage{natbib}
\bibliographystyle{abbrvnat}
%\setcitestyle{authoryear, open={((},close={))}
\renewcommand{\cite}{\citep}


% configure headline

\ihead{Tilman Hinnerichs}
\ohead{Linking Connectivities and Gene Expression Patterns in Mice Brains}
\cfoot*{\pagemark}
%opening
\title{Linking Connectivities and Gene Expression Patterns in Mice Brains}
\author{Tilman Hinnerichs}
\date{}
\pagestyle{headings}

% Build subsubsubsection
\usepackage{titlesec}
\usepackage{todonotes}

%Definitions
\newtheorem{mydef}{Definition}
\newtheorem{example}{Example}
\newtheorem{lemma}{Lemma}
\newtheorem{theorem}{Theorem}
\newtheorem{corollary}{Corollary}
\newtheorem{proposition}{Proposition}

\newcommand{\name}{Treasure}

\newcommand{\HRule}[1]{\rule{\linewidth}{#1}}
\setcounter{tocdepth}{5}
\setcounter{secnumdepth}{5}

%-------------------------------------------------------------------------------
% TITLE PAGE
%-------------------------------------------------------------------------------

\begin{document}
	
\title{ \normalsize \textsc{Diploma thesis}
	\\ [2.0cm]
	\HRule{0.5pt} \\
	\LARGE \textbf{\uppercase{Linking Connectivities and Gene Expression Patterns in Mice Brains}
		\HRule{1pt} \\ [0.5cm]
		\normalsize September 30, 2021 \vspace*{5\baselineskip}}
	
	\date{Summer semester 2021}
	
	\author{
		Tilman Hinnerichs \\
		Matrikelnummer: 4643427 \\ 
		Technische Universität Dresden\vspace{1cm}\\
		Tutor: Dr. Nico Scherf\\
		MPI for CBS}}
\maketitle
\todo{proper title?}
\newpage
\begin{abstract}
	
\end{abstract}

\newpage

\tableofcontents

\newpage


\subsection*{To be sorted somewhere}
\begin{itemize}
	\item Variability and different interpretations of different graph convolutional neural filters \cite{GCNConv, GENConv2020, feng2022kergnns} etc.
	
	
	\item DeepGOPlus for feature generation \cite{DeepGoPlus}
	\item discussion of different PPI network databases \cite{STRINGv10}
	\item discussion of potential databases associating gene expression data with their spatial distribution \cite{hawrylycz_digital_2011}
	\item discussion of best neural learning/graph convolutional methods \cite{Pytorch, PytorchGeometric}
	\item how to handle highly imbalanced data, metrics, preprocessing, sampling, modification of loss function \cite{Jeni2013} and optimization over them (with Adam\cite{Adam2014})
	\item maybe introduction of PhenomeNET for MP/GO for more sophisticated protein representation \cite{PhenomeNET2011, GOoriginal2000, GOrecent2020, MP2009} and derive features from DL2vec \cite{DL2vec2020, Word2vec2013}
	\item evaluation of \glqq Using ontology embeddings for structural inductive bias in gene expression data analysis\grqq{}\cite{Trebacz2020}
	\item take some ideas from \citet{Zitnik2017} with title \glqq Predicting multicellular function through multi-layer tissue networks\grqq{}. (OhmNet)
	\item potentially group results based on InterPro\cite{Interpro2020} families eventually
	
	\item choice of model organism?!
\end{itemize}


\newpage

\section{Introduction}
\label{sec:introduction}
General thread for introduction and motivation:
\begin{itemize}
	\item Gene expression patterns are difficult to analyze in humans $\rightarrow$ take mouse as model organisms
	\item The brain is a multi-level system in which the high-level functions are generated by low-level genetic mechanisms. Thus, elucidating the relationship among multiple brain levels via correlative and predictive analytics is an important area in brain research. Currently, studies in multiple species have indicated that the spatiotemporal gene expression patterns are predictive of brain wiring. Specifically, results on the worm Caenorhabditis elegans have shown that the prediction of neuronal connectivity using gene expression signatures yielded statistically significant results.
	\item no in-depth analysis of mouse brain genetic patterns and their relation to different connectivity patterns has been made yet
	\item we analyze 
	
	\item studies have shown circadian patterns of gene expression in human brain and the disruption of those in depressive disorder \cite{li2013circadian}
	\item \cite{twine2011whole} show the importance of gene expression patterns, by linking gene expression abberation with increase in Alzheimer's disease
	
	\item Guilt by association over gene networks \cite{Oliver2000, Gillis2012} in genetic networks
	\item protein function prediction from PPI networks \cite{Vazquez2003}
	\item GCNs have been applied successfully to variety of tasks over different types of graphs.
\end{itemize}

\subsection*{General Introduction of the Research Study}

\subsection*{Research problem or Questions with Sub-Questions}

\subsection*{Reasons or Needs for the Research Study/Motivation for my research}

\subsection*{Definition and explanation of Key Terminology}

\subsection*{Context of Research Study within th Greater Discipline}

\begin{itemize}
	\item Introduction to mouse brains as model organisms for insights into human brain
	\item Works on mouse brain in general and potential tasks
	\item works on gene expression in mouse brains
	\begin{itemize}
		\item traditional approaches
		\item importance of gene expression patterns in mouse brains
	\end{itemize}
	\item neural networks for this purpose
	\begin{itemize}
		\item how were 
	\end{itemize}
	\item gene expression for general tissue
	
\end{itemize}

\newpage
\section{Literature overview}
\label{sec:relatedwork}
\subsection{Gene expression databases and prediction}
\label{sec:relatedwork_geneexp}
Research in gene expression prediction and profiling has a long history in bioinformatics and systems biology, but was almost exclusively linked to cancer research. Moreover, with the rise of machine learning, and more specifically (deep) neural networks and its variants, this field became increasingly data reliant. The Human Genome Project \cite{watson1990human}, launched in 1990 and declared finished in 2003 while the first gapless assembly was finished in 2022, also sparked various works in relating these genetic representations to other tissue- and individual-specific properties and traits.\\ 

For comparison of gene expression profiling works there exist multiple prominent variables. Most significantly, the chosen organism is a crucial choice for both data availability and predictive complexity. Second, the chosen tissue is naturally important for the proposed hypotheses, especially with respect to tissue definitive cancer research, and its potential ability to generalize without transfer learning. While gene expression pattern analysis approaches frequently focus on tissues like \textit{mamma}, (primarily female) breast, \cite{herschkowitz2007identification}, liver \cite{flores2002patterns}, and skeletal muscle \cite{lecker2004multiple} for exploration of diseases like cancer and atrophy, respectively, in humans. 

However, the nervous system is often investigated separately as it bears different molecular processes and structure, anatomy and cell life cycles, while brain and spinal cord are even based in a separate nutritional circuit for mammals. Moreover, gene expression determination in the human brain may almost certainly remain an deadly intervention for most brain tissues, hence allowing only for careful extraction of specific tissues in living organisms. Also this disallows for \textit{in-vivo} extraction of vital brain regions and structures, e.g. the brainstem. Furthermore, the human brain's gene expression patterns are varied and diversified \cite{ramasamy2014genetic}, aligning with its anatomical and embryogenesis complexity, and its compartments are exceptionally and deeply connective and collaborative \cite{fornito2015connectomics}. Both also hold for invertebrates, i.e. insects. Thus, full genetic profiles of expression are mandatory for a full understanding of the mammalian and invertebrate brain and primary nervous system, respectively. By the strong intervention of the tissue extraction, full genome atlases are fit together from various experiments on multiple individuals.

The human brain is among the most intricate and complicated networks we do know of, and is far from being fully understood. Additionally, full transcriptomic atlases of human brains are difficult to collect while raising decisive privacy concerns. Yet, there were multiple efforts and projects with rather small sample sizes. A detailed elaboration on dataset and organism choice, and their respective properties may be found in Section \ref{sec:datasets}. \\

However, there have been works on numerous works for other tissues and other organisms. \citet{modencode2010identification} correlate activity patterns in the regulatory network within \textit{Drosophile}, proposing their model for identification of functional elements "modENCODE". As this work was published back in \citeyear{modencode2010identification}, the approach relies purely on statistical correlation and covariance. \citet{chikina2009global} follow a similar approach in \textit{C. elegans} predicting tissue-specific gene expression in \citeyear{noble2006support} utilizing support-vector machines (SVM)\cite{noble2006support}. 

More modern, data-oriented machine learning models such as (deep) neural networks (NN) were applied successfully to similar problems. \citet{aromolaran2020essential} achieved to predict essential genes based on their respective sequence and functional features profiting off NNs, while transcriptomic interaction prediction was done based on functional gene data using deep learning in \citet{yang2019predicting} in \textit{Drosophila} over different tissues. 

Within humans, as mentioned previously, gene expression was primarily used for cancer and disease research. \citet{schulte2021integration} and \citet{wang2021mogonet} were the first to apply graph convolutional neural networks to the task of gene expression prediction within humans. While \citet{schulte2021integration} was applied on data from The Cancer Genome Atlas (TCGA)\cite{tomczak2015review} across multiple tissues, \citet{wang2021mogonet}'s MOGONET is proposed as a general framework for gene expression prediction with example computations on ROSMAP dataset and TCGA. Both approaches implement the original formulation of GCNs\cite{GCNConv}, which we will discuss in more detail in Section \ref{sec:graphconv}, over protein-protein interaction networks and accomplish outstanding performances and both measure biomarker importance for prediction in order to leverage explainability. The authors thereby exploit the "guilt by association" principle \cite{Oliver2000, Gillis2012} over gene networks, adding background knowledge such as biological interaction and pathways.

Crucial for almost all classification tasks in machine learning is the choice of entity representation. In the mentioned works molecular \cite{schulte2021integration, modencode2010identification, noble2006support} and phenotypical \cite{wang2021mogonet, chikina2009global} features were used for expression prediction, but never both combined. The combination of phenotypical and molecular features over GCNs was proven to raise predictive performance in drug-target interaction prediction \cite{hinnerichs2021dti} but remains an open challenge for this very task.


\subsection{Finding spatial patterns in gene expression in mice brains}
\label{sec:relatedwork_micebrains}
In this subsection we will constrain the issue of gene expression analysis to both "spatial patterns", mammals and the tissues of the brain, which we study in this work. The term \textit{spatial patterns} is rather vague and allows for various interpretations, both discrete and continuous, which will form the classes for the following literature review.\\

\citet{zapala2005adult} is among the earliest works, showing that local structures beared "transcriptional imprint" that coincide with the embryological origin of the examined regions. However, they only were able to identify up to 24 neural tissues. They further conclude that this may be important for functional collaboration within the adult mouse brain. The authors measure pairwise correlation show the existence of clusters over a heatmap. \\

The Allen Institute Brain Atlas (AIBA), is a collection various atlases such as Allen Mouse Brain Atlas (AMBA)\cite{MouseBrainAtlas, daigle2018suite}, the Allen Mouse Brain Connectivity Atlas (AMBCA) \cite{oh2014mesoscale, harris2019hierarchical} and the Allen Mouse Brain Common Coordinate Framework (CCFv3) \cite{wang2020allen} to name only the ones related to adult mice's brains. As it was the first coherent collection of spatially resolved expression values, mapping 2D expression images consistently to 3D coordinates, the AMBA has sparked a range of publications. Within \citet{MouseBrainAtlas}, the Allen Institute also published the "Allen Reference Atlas"(ARA) proposing a number morphological and histologically induced sub-regions of the brain and hence a precisely defined parcelation. Moreover, they propose the ARA \textit{ontology}, a semnatic hierarchy, providing a hierarchical cluster of all sub-structures and map them back to their coordinates with the CCF. 

\todo{Name a few sparked research works on this}

\citet{takata_flexible_2021} propose a flexible annotation atlas of the mouse brain, introducing a flexible ontology construction framework which may be used on the transcriptomic data such as the AMBA, leveraging anatomic structure and axonal projection data. Here, FAA focuses on consistent and reproducible regions-of-interest (ROIs) definition for other downstream tasks such as resting-state functional connectivity annotation. Further, this ontology may be seen as a pattern within mouse brain, while it may only detect connected structures. 

The authors of \citet{ValkShapingBrainStructure2020} analyze structural covariance of cortical thickness within primate brains, namely macaques, and its correlation to each cortical layers transcriptome. Further, transcriptomic variation was related to a continuum of functions by mapping them the brain anatomy, inducing a \textit{continuous}, functional parcelation of the primates brain. Further, this study suggests a relation of functional and transcriptomic links.

While also focused on mouse brains, \citet{Partel2020} submits a novel database based on their own \textit{in-situ} sequencing data, and a consecutive spatial gene expression analysis pipeline, and relates the results to tissue morphology and hence indirectly to the AMBA. Similarly to our proposed approach, brain parcelation are present as $n$-dimensional, continuous embeddings, representing closeness in gene expression space.  Due to the similarity in the pipeline especially in their visualization utilizing UMAP, we will use the generated images of this work for a brief comparison in Section \ref{sec:results}.



\begin{itemize}

	\item Sparked by the \citet{MouseBrainAtlas}
	\item 
\end{itemize}
\begin{itemize}
	\item Have GCNs be applied before here?
	\item usage of ontologies?
	\begin{itemize}
		\item functional graph 
		\item structural ontology?
		\item developmental ontology
	\end{itemize}
\end{itemize}


\subsection{Structural and functional connectivity prediction}

\begin{itemize}
	\item \cite{lee2020prediction} train classifiers using blood gene expression data in order to predict Alheimer's disease in Humans
	\item \cite{fornito2015connectomics} elaborate on the connectomics of brain disorders and its complexity in connectivity. Understanding how brain networks respond to pathological perturbations is crucial for understanding brain disorders and behavior
\end{itemize}
\subsection*{Structural/Axonal connectivity}

\begin{itemize}
	\item \cite{fakhry2015high} predict axonal connectivity from gene expression patterns in mice brain with an accuracy of 93\%
	\item \cite{roberti2019exploiting} use transcriptomic information to anatomical connectivity patterns and gene expression of neurons using (shallow) neural networks. Yield a 85\% accuracy in prediction of unconnected and connected regions.
	
\end{itemize}

\begin{itemize}
	\item experimental setup from Allen Institute for axonal projection data
	\item paraphrase description of \glqq Technical tour: Explore the Allen Mouse Brain Connectivity Atlas\grqq{}
\end{itemize}



\subsection*{Functional connectivity}
\begin{itemize}
	\item \cite{whitfield2003gene} were one of the first to link transcriptomic data with behavior and hence functional patterns in individual honey bees back in 2003. The authors show that changes in the messenger RNA were connected to behavior.
	\item \cite{rankin2002gene} first developed the idea of combining behavioral analyses of \textit{Caenorhabditis elegans} with their genetics. Further, \cite{sun2021temporal} only recently described the distinct functional states and the corresponding distinct molecular states within the transcriptome. While honey bees and nematodes are rather simple model organisms, enabling both full transcriptomic analyses of the organisms, and their bearing and actions. However, "behavior" may be ambiguous and vague for such taxonomically distant animals, from the viewpoint of humans, and may only be linked to very basic meta-tasks such as basic routing, orientation and basic social interaction. 
	\item Why do we not directly investigate gene expression patterns in human brains?
	\begin{itemize}
		\item only few data points are given for the entirety of the human brain, while spatial decomposition and partitioning is crucially more complex. 
	\end{itemize}
	\item \cite{wang2022network} recently proposed a novel-network based method integrating molecular-based gene association networks such as protein-protein interaction networks with brain connectome data. They further link these gene expression patterns to four brain diseases, including Alzheimer’s disease, Parkinson’s disease, major depressive disorder and autism.
\end{itemize}
\begin{itemize}
	\item Where is data coming from? \cite{AIDAmri2019}
	\item How to calculate functional connectivity matrix $\rightarrow$ AIDAconnect (no paper yet? cite dataset?
	\item How to combine functional connectivity for multiple samples? 
	\item \cite{Zerbi2021}
\end{itemize}


\subsection*{Brief Overview of LIterature Reviewed, Discussed and applied}

\subsection*{Study Model and Process Aligning with literature reviewed}

\subsection*{Hypotheses and justifications tied to prior sections and statements}

\subsection*{The Scope of the study with theoretical assumptions and limitations}

\subsection*{To be searched}
\begin{itemize}

	\item find other papers on
	\begin{itemize}
		\item gene expression patterns within mouse brain and both possible hypothesis and tasks, and models over this
		\item gene knockout models and whether they can learn propagation of those?
		\item connection of FC and gene expression patterns and how to prove such interaction/correlation?
		\item possible gene knockout targets within mouse brain and possible structural influences
	\end{itemize}
\end{itemize}



\subsection*{Spatial patterns of gene expression}

Data discussion, hypotheses and traditional approaches:
\begin{itemize}
	\item \cite{noauthor_clustering_nodate}
	\item Possible effects of rabies virus on gene expression\cite{prosniak_effect_2001} for potential knockout targets
	
	\item Review paper on regional variation in gene expression in mouse brain \cite{pavlidis_analysis_2001}
\end{itemize}

Modern approaches on learning from gene expression patterns in mouse brain:
\begin{itemize}
	\item Deep learning methods for capturing spatiality w.r.t. gene expression withing the brain \cite{zeng_deep_2015}
	\item R package for simulating gene expression from graph structures over general biological pathways \cite{kelly_graphsim_2020} \todo[inline]{Read this}
\end{itemize}



\newpage
\section{Materials and methods}
\label{sec:methods}
In this study, we utilized and incorporated various approaches from other works and applied them to diverse datasets. The following section will give a brief overview over all modules of the proposed model, while the combined method will be presented and described in the results section (Section \ref{sec:results}).
\subsection*{Introduction and general description, study method and study design}
\subsection{Problem description}
\label{sec:probdesc}
Here we give a brief introduction to each of the three tackled issues and further summarize data properties, challenges and goals of each problem. 
\subsubsection{Spatial gene expression prediction}
Firstly, the issue of gene expression prediction 

\subsubsection{Dimensionality reduction in brains}
\begin{itemize}
	\item 
\end{itemize}

\subsubsection{Structural and functional connectivity prediction}
\begin{itemize}
	\item 
\end{itemize}


\subsection*{Assumptions of study method and study design with implied }

\subsection{Datasets}
\label{sec:datasets}
As human brains are among the most complex in structure and connectivity within nature, a full transcriptomic atlas may be very valuable for the research community and our experiments in this work. However, full transcriptomic atlases of homo sapiens are ethically difficult to gather. Additionally, as a valuable, public genetic atlas of deceased relatives may provide highly critical information about the remaining, living ones, such as genetic diseases, genetic markers for correlating with addiction and other social behavior, or ancestry in general, this raises tremendous privacy concerns. As we want to investigate transcriptomic patterns in the brain and their relation to structural and functional connectivity as a generalized, organism-invariant methodology, we also want our experiments to be as understandable and replicable as possible. As rodents and more specifically mice are well studied in behavior and due to their taxonomic proximity to humans serve as model organisms for diverse genetic, social and medical experiments, we opted for mice as the study organism. The ultimate goal still shall be the further understanding of brains of our species.

However, there have been multiple initiatives towards collaborative and open human brain data, such as the 
\begin{itemize}
	\item Human connectome atlas
	\item Allen Human Brain atlas -> very small sample size
\end{itemize}
Furthermore, immense effort was put into enormous projects and databases for example invertebrates, namely \textit{Drosophila} (specifically \textit{Drosophila melanogaster}, also called \textit{fruit fly}) and \textit{Caenorhabditis elegans} (short: \textit{"C. elegans"}, colloquially also called \textit{roundworm}) with the two projects "Virtual Fly Brain" \cite{milyaev2012virtual} (\href{https://virtualflybrain.org/}{https://virtualflybrain.org/}) and "Wormbase" \cite{lee2003building, davis2022wormbase} (\href{https://wormbase.org}{https://wormbase.org}), respectively. Yet, we wanted to stay within the same taxonomic phylum leading our choice towards mice brains. 

In literature there are several approaches, compromising sequencing depth, accuracy, throughput and spatial resolution. Generally, one can identify two classes of gene expression measurement while preserving spatial information. The first approach is to store spacial coordinates first, followed by a the sequencing of single-cell RNA where \citet{achim2015high} and \citet{chen2017spatial} propose the mapping and the Geo-seq protocol for this method. The second method includes the usage of "barcodes", decoded in the tissue sample, while running a parallel analysis of numerous mRNAs \cite{ke2013situ, moffitt2016high}. Samples are collected from multiple individuals.


\subsubsection{Spatial gene expression values in mice brain}
\begin{itemize}
	\item include average brain and annotation brain for parcelation
	\item copy introduction from graph-iss
	\item add section on gene set enrichment analysis (GSEA)
\end{itemize}

Protein-protein interaction graph:
\begin{itemize}
	\item CPDB78, STRING-db79, IRefIndex80, Multinet81 and PCNet82
	\item STRING\cite{STRINGv10} for mice
\end{itemize}

\subsubsection{Structural and functional connectivity databases}


\subsection*{in-depth description of the study design/datasets used and motivation why they were used for these experiments}
\subsection*{Explanation of Sample used in the study}
\begin{itemize}
	\item show distribution (histo, mean, median, boxplot?) of expression densities see `get\_ge\_structure\_mat`

	\item how to normalize expression intensity (see discussion in DeepMOCCA paper, Sara Alghamdi), as there are regions with much more activity than others (e.g. bone narrow vs. bone boarder); thresholds for intensity varies across genes
	
	\begin{itemize}
		\item over all intensities $\rightarrow$
		\item per structure $\rightarrow$  
		\item per gene $\rightarrow$ 
	\end{itemize}
\end{itemize}

\begin{itemize}
	\item why were these datasets used and not others?
	\item How did we achieve the matching?
	\item what are premises of the dataset?
	
	\item transfer learning working for other structure/regions
	\item dataset: Allen Mouse brain atlas vs. 
	\begin{itemize}
		\item \href{https://www.har.mrc.ac.uk/harwell-news/phenoview-new-tool-compare-impc-data/}{phenoview impc data}
		
		\item \href{https://www.mousephenotype.org/}{mousephenotype}
		
		\item \href{http://www.informatics.jax.org/expression.shtml}{HPO/MP project expression data}
	\end{itemize}
	\item Mouse brain CCFv3
\end{itemize}
\begin{itemize}
	\item Allen mouse brain atlas \cite{MouseBrainAtlas}
	\begin{itemize}
		\item discussion on different normalization schemes
	\end{itemize}
	\item STRING for PPI network and how we chose suitable interactions \cite{STRINGv10}
\end{itemize}

Four graphs were used in this study:
\begin{itemize}
	\item Protein-protein interaction graph from STRING
	\item structure hierarchy/ontology from \cite{MouseBrainAtlas}
	\item structural connectivity data from (Mouse Projection data)
	\item functional connectivity data from \cite{AIDAmri2019}
\end{itemize}

\subsection{Model}
\label{sec:modeldesc}
\subsection*{Explanation of Measurement, Definitions, Indexes, Reliabililty and Validity of study method and study design}
\subsection*{Description of Analytical Tehcniques to be Applied and justification for them}

\subsection*{Reliability and validity of internal/external design and related subtypes}

\subsubsection{Feature generation}

Data preparation for regression task
\begin{itemize}
	\item unbalanced data for prediction task
\end{itemize}

\subsubsection{Graph convolutional neural layers}
\label{sec:graphconv}
We include these molecular and ontology-based sub-models within a
graph neural network (GNN) \cite{GCNConv}. The graph underlying the GNN is
based on the protein--protein interaction (PPI) graph. The PPI dataset
is represented by a graph $G=(V,E)$, where each protein is represented
by a vertex $v\in V$, and each edge $e\in E\subseteq V\times V$
represents an interaction between two proteins. Additionally, we
introduce a mapping $x:V\rightarrow\mathbb{R}^{d}$ projecting each
vertex $v$ to its node feature $x_v := x(v)$, where $d$ denotes the
dimensionality of the node features.

% As described before, graph convolution has shown significant
% performance increase in a variety of tasks. While there are various
% methods out there we will only introduce the most basic one here. 
A graph convolutional layer \cite{GCNConv} consists of a learnable
weight matrix followed by an aggregation step, formalized by
\begin{equation}
	\mathbf{X}^{\prime} = \mathbf{\hat{D}}^{-1/2} \mathbf{\hat{A}}
	\mathbf{\hat{D}}^{-1/2} \mathbf{X} \mathbf{\Theta}
\end{equation}
where for a given graph $G=(V,E)$, $\hat{A} = A + I$ denotes the
adjacency matrix with added self-loops for each vertex, $D$ is
described by $\hat{D}_{ii} = \sum_{j=0} \hat{A}_{ij}$, a diagonal
matrix displaying the degree of each node, and $\Theta$ denotes the
learnable weight matrix. Added self-loops enforce that each node
representation is directly dependent on its own preceding one. The
number of graph convolutional layers stacked equals the radius of
relevant nodes for each vertex within the graph.

The update rule for each node is given by a message passing scheme
formalized by
\begin{equation}
	\mathbf{x}^{\prime}_i = \mathbf{\Theta} \sum^{N}_{j}
	\frac{1}{\sqrt{\hat{d}_j \hat{d}_i}} \mathbf{x}_j
\end{equation}
where both $\hat{d}_i, \hat{d}_j$ are dependent on the edge weights
$e_{ij}$ of the graph. With simple, single-valued edge weights such as
$e_{ij}=1 \text{ }\forall (i,j)\in E$, all $\hat{d}_i$ reduce to
$d_i$, i.e., the degree of each vertex $i$. We denote this type of
graph convolutional neural layers with \textsc{GCNConv}.

While in this initial formulation of a GCNConv the node-wise update
step is defined by the sum over all neighboring node representations,
we can alter this formulation to other message passing schemes.  We
can rearrange the order of activation function $\sigma$, aggregation
$\mathrm{AGG}$, and linear neural layer $\mathrm{MLP}$ with this
formulation as proposed by \cite{GENConv2020}:
\begin{equation}
	\mathbf{x}_i^{\prime} = \mathrm{MLP} \left( \mathbf{x}_i +
	\mathrm{AGG} \left( \left\{
	\mathrm{\sigma} \left( \mathbf{x}_j + \mathbf{e_{ji}} \right) +\epsilon
	: j \in \mathcal{N}(i) \right\} \right)
	\right)
\end{equation}
where we only consider
$\sigma \in \{\mathrm{ReLU}, \mathrm{LeakyReLU}\}$. We denote this
generalized layer type as \textsc{GENConv} following the notation of
PyTorch Geometric \cite{PytorchGeometric}.  While the reordering is
mainly important for numerical stability, this alteration also addresses
the vanishing gradient problem for deeper convolutional networks
\cite{GENConv2020}. Additionally, we can also generalize the
aggregation function to allow different weighting functions such as
learnable $\mathrm{SoftMax}$ or $\mathrm{Power}$ for the incoming
signals for each vertex, substituting the averaging step in
\textsc{GCNConv}. Hence, while \textsc{GCNConv} suffers from both
vanishing gradients and signal fading for large scale and highly
connected graphs, each propagation step in \textsc{GENConv} emphasizes
signals with values close to $0$ and $1$. The same convolutional
filter and weight matrix are applied to and learned for all nodes
simultaneously. % , and the resulting information\todo{Which information?
% Specify} hold no information on their own connectivity.
We further employ another mechanism to avoid redundancy and fading
signals in stacked graph convolutional networks, using residual
connections and a normalization scheme \cite{DeepGCN2019}
	\cite{GENConv2020} as shown in Supplementary 3.  The residual
blocks are reusable and can be stacked multiple times.

\begin{itemize}
	\item what is GATConv?
	\item what is KerGNN and what is its idea?
	\item add some sentences to the section above
	\item node vs. graph classification vs. link prediction
\end{itemize}

\subsubsection{Dimensionality reduction techniques}
\paragraph{Principal component analysis (PCA)}
\label{sec:pca}

\paragraph{tSNE}
\label{sec:tsne}

\paragraph{UMAP}
\label{sec:umap}

\paragraph{Parametric UMAP}
\label{sec:paraumap}


\subsubsection{Hyperparameter tuning}
\begin{itemize}
	\item RayTune\cite{liaw2018tune} for automated hyperparameter tuning
\end{itemize}

\subsection{Evaluation and metrics}
\label{sec:evalmetrics}
\begin{itemize}
	\item AUC and AUPR for gene expression prediction
	\item self-built metric for evaluation of dim-red
	\item AUC and AUPR for conn pred
\end{itemize}


\newpage
\section{Results}
\label{sec:results}
\subsection{Gene expression prediction}

\begin{itemize}
	\item We originally started from the per section prediction in order to paste its performance and results to other "related" structures within in the mouse brain. We propose multiple ideas \dots. As mentioned we used three different feature types in this study. \dots (molecular features, phenotypical features, pure taxonomic features (InterPro embedding))) \dots Due to the poor performance of the predictor with all three used feature types, we abandoned these plane
	\begin{itemize}
		\item 
	\end{itemize}
	\item structure specific features?
	\begin{itemize}
		\item structural ontology / closeness
		\item developmental hierarchy of tissue
	\end{itemize}
\end{itemize}

Our model also allows us to test different ways of representing omics data. We
tested different ways to normalize values assigned to genes as these normalizations
convey different biological information; in the matrix of values assigned to genes from
cancer samples, we can normalize values across the entire matrix, across each row
(cancer sample), or across each column (gene). While a global normalization is more
common, row-based normalization allows us to highlight values that are significantly
higher or lower within one sample (e.g., which genes are expressed at high or low levels within a single sample), and column-based normalization allows us to highlight values
assigned to a particular gene that are significantly higher or lower within one sample
(e.g., whether a gene is expressed at higher or lower levels within one sample compared
to all others). We find that column-based normalization performs better than row-based
normalization, while the global normalization approach performs close to random. The
best results are achieved when combining both row- and column-based normalization
(Supplementary Table 2).	

\subsection{Dimensionality reduction and its combination with different graphs structures}

\begin{itemize}
	\item plot for showing validity of embeddings: K-means colour with respect to cluster 
	\item plot colour parent structure all similar
\end{itemize}

\subsection{On the linkage of connectivities and gene expression patterns}

\subsection*{Brief Overview of Material}
\subsection*{Findings (Results) of the Method of Study and Any Unplanned or Unexpected Situations that Occurred}
\subsection*{Brief Descriptive Analysis
Reliability and Validity of the Analysis}
\subsection*{Explanation of the Hypothesis and Precise and Exact Data (Do Not Give Your Opinion)}


\newpage
\section{Discussion}
\label{sec:discussion}
\subsection*{Brief Overview of Material}
\subsection*{Full Discussion of Findings (Results) and Implications}
\subsection*{Full Discussion of Research Analysis of Findings}
\subsection*{Full Discussion of Hypothesis and of Findings}
\subsection*{Post Analysis and Implications of Hypothesis and of Findings}

Novelty:
\begin{itemize}
	\item GCNs over gene expression was never applied here
\end{itemize}



\newpage
\section{Conclusion}
\label{sec:conclusion}

\subsection*{Summary of Academic Study}
\subsection*{Reference to Literature Review}
\subsection*{Implications of Academic Study}
\subsection*{Limitations of the Theory or Method of Research}
\subsection*{Recommendations or Suggestions of Future Academic Study}

\begin{itemize}
	\item gene expression patterns within mouse brain and both possible hypothesis and tasks, and models over this
	\item gene knockout models and whether they can learn propagation of those?
	\item connection of FC and gene expression patterns and how to prove such interaction/correlation?
	\item possible gene knockout targets within mouse brain and possible structural influences
\end{itemize}


\newpage

\bibliography{citations}

\end{document}