\documentclass[]{article}
\usepackage[left=3cm,right=3cm,top=1.5cm,bottom=2cm,includeheadfoot]{geometry} 
\usepackage{babel}
\usepackage{hyperref}
\usepackage{mwe}
\usepackage[markcase=noupper
]{scrlayer-scrpage}
\usepackage{amsmath}
\usepackage{amssymb}
\usepackage{multirow}

% bibliography natbib
\usepackage{natbib}
\bibliographystyle{abbrvnat}
%\setcitestyle{authoryear, open={((},close={))}
\renewcommand{\cite}{\citep}


% configure headline
\KOMAoptions{
	headsepline = true
}
\ihead{Tilman Hinnerichs}
\ohead{Towards gene expression prediction in mouse brains}
\cfoot*{\pagemark}
%opening
\title{Towards gene expression prediction in mouse brains}
\author{Tilman Hinnerichs\\
	\href{mailto:tilman@hinnerichs.com}{tilman@hinnerichs.com}}
\date{}
\pagestyle{headings}

% Build subsubsubsection
\usepackage{titlesec}
\usepackage{todonotes}

%Definitions
\newtheorem{mydef}{Definition}
\newtheorem{example}{Example}
\newtheorem{lemma}{Lemma}
\newtheorem{theorem}{Theorem}
\newtheorem{corollary}{Corollary}
\newtheorem{proposition}{Proposition}

\newcommand{\name}{Treasure}

\newcommand{\HRule}[1]{\rule{\linewidth}{#1}}
\setcounter{tocdepth}{5}
\setcounter{secnumdepth}{5}

%-------------------------------------------------------------------------------
% TITLE PAGE
%-------------------------------------------------------------------------------

\begin{document}
	
\title{ \normalsize \textsc{Diploma thesis}
	\\ [2.0cm]
	\HRule{0.5pt} \\
	\LARGE \textbf{\uppercase{Towards gene expression prediction in mouse brains}
		\HRule{1pt} \\ [0.5cm]
		\normalsize September 30, 2021 \vspace*{5\baselineskip}}
	
	\date{Summer semester 2021}
	
	\author{
		Tilman Hinnerichs \\
		Matrikelnummer: 4643427 \\ 
		Technische Universität Dresden\vspace{1cm}\\
		Tutor: Dr. Nico Scherf\\
		MPI for CBS }}
\maketitle
\todo{proper title?}
\newpage
\begin{abstract}
	
\end{abstract}

\newpage

\tableofcontents

\newpage

\section*{To include in some chapter}
Predict gene expression per section/structure:\\
Take region as input and predict gene expression\\
Challenges:
\begin{itemize}
	\item how to normalize expression intensity (see discussion in DeepMOCCA paper, Sara Alghamdi), as there are regions with much more activity than others (e.g. bone narrow vs. bone boarder); thresholds for intensity varies across genes
	
	\begin{itemize}
		\item over all intensities $\rightarrow$
		\item per structure $\rightarrow$  
		\item per gene $\rightarrow$ \\
		
	Our model also allows us to test different ways of representing omics data. We
	tested different ways to normalize values assigned to genes as these normalizations
	convey different biological information; in the matrix of values assigned to genes from
	cancer samples, we can normalize values across the entire matrix, across each row
	(cancer sample), or across each column (gene). While a global normalization is more
	common, row-based normalization allows us to highlight values that are significantly
	higher or lower within one sample (e.g., which genes are expressed at high or low levels within a single sample), and column-based normalization allows us to highlight values
	assigned to a particular gene that are significantly higher or lower within one sample
	(e.g., whether a gene is expressed at higher or lower levels within one sample compared
	to all others). We find that column-based normalization performs better than row-based
	normalization, while the global normalization approach performs close to random. The
	best results are achieved when combining both row- and column-based normalization
	(Supplementary Table 2).	
	\end{itemize}
	\item transfer learning working for other structure/regions
	\item dataset: Allen Mouse brain atlas vs. 
	\begin{itemize}
		\item \href{https://www.har.mrc.ac.uk/harwell-news/phenoview-new-tool-compare-impc-data/}{phenoview impc data}
		
		\item \href{https://www.mousephenotype.org/}{mousephenotype}
		
		\item \href{http://www.informatics.jax.org/expression.shtml}{HPO/MP project expression data}
	\end{itemize}
	\item structure specific features?
	\begin{itemize}
		\item structural ontology / closeness
		\item developmental hierarchy of tissue
	\end{itemize}
\end{itemize}

\subsection*{Possible hypotheses}
\begin{itemize}
	\item predict gene expression for a given single structure
	\item predict structure from gene expression pattern
	\item predict structure form gene expression and image
	\item predict cancer type from morphology/pathologic image of cancer
	\item simulate loss of function/expression by removing one node of graph
\end{itemize}

\subsection*{Ideas for page-filling plots}
\begin{itemize}
	\item show distribution (histo, mean, median, boxplot?) of expression densities see `get\_ge\_structure\_mat`
	\item show predictability of gene expression across over multiple structure vs. distance over onto?
\end{itemize}

\section{Introduction}

\begin{itemize}
	\item Works on mouse brain in general and potential tasks
	\item works on gene expression in mouse brains
	\item conservative approaches
	\item neural networks for this purpose
	\item gene expression for general tissue
	
\end{itemize}

\section{Literature review}

\begin{itemize}
	\item Variability and different interpretations of different graph convolutional neural filters \cite{GCNConv, GENConv2020, SAGEConv} etc.
	\item Guilt by association over gene networks \cite{Oliver2000, Gillis2012}
	\item protein function prediction from PPI networks \cite{Vazquez2003}
	\item DeepGOPlus for feature generation \cite{DeepGoPlus}
	\item discussion of DeepMocca by Sara \cite{DeepMocca2021}
	\item discussion of different PPI network databases \cite{STRINGv10}
	\item discussion of best neural learning/graph convolutional methods \cite{Pytorch, PytorchGeometric}
	\item how to handle highly imbalanced data, metrics, preprocessing, sampling, modification of loss function \cite{Jeni2013} and optimization over them (with Adam\cite{Adam2014})
	\item maybe introduction of PhenomeNET for MP/GO for more sophisticated protein representation \cite{PhenomeNET2011, GOoriginal2000, GOrecent2020, MP2009} and derive features from DL2vec \cite{DL2vec2020, Word2vec2013}
	\item evaluation of \glqq Using ontology embeddings for structural inductive bias in gene expression data analysis\grqq{}\cite{Trebacz2020}
	\item take some ideas from \citet{Zitnik2017} with title \glqq Predicting multicellular function through multi-layer tissue networks\grqq{}. (OhmNet)
	\item potentially group results based on InterPro\cite{Interpro2020} families eventually
	\item RayTune\cite{liaw2018tune} for automated hyperparameter tuning
	
	
	
\end{itemize}

\section{Methods}
\subsection{Problem description}

\subsection{Datasets}
\begin{itemize}
	\item Allen mouse brain atlas \cite{MouseBrainAtlas}
	\item STRING for PPI network and how we chose suitable interactions \cite{STRINGv10}
\end{itemize}

\subsection{Model}

\subsubsection{Feature generation}

\subsubsection{Graph convolutional neural layers}

We include these molecular and ontology-based sub-models within a
graph neural network (GNN) \cite{GCNConv}. The graph underlying the GNN is
based on the protein--protein interaction (PPI) graph. The PPI dataset
is represented by a graph $G=(V,E)$, where each protein is represented
by a vertex $v\in V$, and each edge $e\in E\subseteq V\times V$
represents an interaction between two proteins. Additionally, we
introduce a mapping $x:V\rightarrow\mathbb{R}^{d}$ projecting each
vertex $v$ to its node feature $x_v := x(v)$, where $d$ denotes the
dimensionality of the node features.

% As described before, graph convolution has shown significant
% performance increase in a variety of tasks. While there are various
% methods out there we will only introduce the most basic one here. 
A graph convolutional layer \cite{GCNConv} consists of a learnable
weight matrix followed by an aggregation step, formalized by
\begin{equation}
	\mathbf{X}^{\prime} = \mathbf{\hat{D}}^{-1/2} \mathbf{\hat{A}}
	\mathbf{\hat{D}}^{-1/2} \mathbf{X} \mathbf{\Theta}
\end{equation}
where for a given graph $G=(V,E)$, $\hat{A} = A + I$ denotes the
adjacency matrix with added self-loops for each vertex, $D$ is
described by $\hat{D}_{ii} = \sum_{j=0} \hat{A}_{ij}$, a diagonal
matrix displaying the degree of each node, and $\Theta$ denotes the
learnable weight matrix. Added self-loops enforce that each node
representation is directly dependent on its own preceding one. The
number of graph convolutional layers stacked equals the radius of
relevant nodes for each vertex within the graph.

The update rule for each node is given by a message passing scheme
formalized by
\begin{equation}
	\mathbf{x}^{\prime}_i = \mathbf{\Theta} \sum^{N}_{j}
	\frac{1}{\sqrt{\hat{d}_j \hat{d}_i}} \mathbf{x}_j
\end{equation}
where both $\hat{d}_i, \hat{d}_j$ are dependent on the edge weights
$e_{ij}$ of the graph. With simple, single-valued edge weights such as
$e_{ij}=1 \text{ }\forall (i,j)\in E$, all $\hat{d}_i$ reduce to
$d_i$, i.e., the degree of each vertex $i$. We denote this type of
graph convolutional neural layers with \textsc{GCNConv}.

While in this initial formulation of a GCNConv the node-wise update
step is defined by the sum over all neighboring node representations,
we can alter this formulation to other message passing schemes.  We
can rearrange the order of activation function $\sigma$, aggregation
$\mathrm{AGG}$, and linear neural layer $\mathrm{MLP}$ with this
formulation as proposed by \cite{GENConv2020}:
\begin{equation}
	\mathbf{x}_i^{\prime} = \mathrm{MLP} \left( \mathbf{x}_i +
	\mathrm{AGG} \left( \left\{
	\mathrm{\sigma} \left( \mathbf{x}_j + \mathbf{e_{ji}} \right) +\epsilon
	: j \in \mathcal{N}(i) \right\} \right)
	\right)
\end{equation}
where we only consider
$\sigma \in \{\mathrm{ReLU}, \mathrm{LeakyReLU}\}$. We denote this
generalized layer type as \textsc{GENConv} following the notation of
PyTorch Geometric \cite{PytorchGeometric}.  While the reordering is
mainly important for numerical stability, this alteration also addresses
the vanishing gradient problem for deeper convolutional networks
\cite{GENConv2020}. Additionally, we can also generalize the
aggregation function to allow different weighting functions such as
learnable $\mathrm{SoftMax}$ or $\mathrm{Power}$ for the incoming
signals for each vertex, substituting the averaging step in
\textsc{GCNConv}. Hence, while \textsc{GCNConv} suffers from both
vanishing gradients and signal fading for large scale and highly
connected graphs, each propagation step in \textsc{GENConv} emphasizes
signals with values close to $0$ and $1$. The same convolutional
filter and weight matrix are applied to and learned for all nodes
simultaneously. % , and the resulting information\todo{Which information?
% Specify} hold no information on their own connectivity.
We further employ another mechanism to avoid redundancy and fading
signals in stacked graph convolutional networks, using residual
connections and a normalization scheme \cite{DeepGCN2019}
	\cite{GENConv2020} as shown in Supplementary 3.  The residual
blocks are reusable and can be stacked multiple times.

\subsubsection{Combined prediction model}

\subsubsection{Hyperparameter tuning}

\subsection{Evaluation and metrics}

\section{Results}

\section{Discussion}

\section{Conclusion}

\newpage

\bibliography{citations}

\end{document}